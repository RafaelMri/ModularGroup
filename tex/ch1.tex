\chapter{Introduction}

\section{M�bius transformations}

In the following we define M�bius transformations and collect some basic properties which will be needed later on.

\begin{definition}
\index{Mobius transformation@M�bius transformation}
A non-constant rational function $\phi \in \C(z)$ of the form 
\begin{equation*}
\phi(z) = \moebius{a}{b}{c}{d}{z},\quad a, b, c, d \in \C,\quad ad - bc \ne 0
\end{equation*}
is called \emph{M�bius transformation}.
\end{definition}

\begin{remark}
The condition $ad - bc \ne 0$ just ensures that $\phi$ is in fact non-constant.
\end{remark}

\begin{theorem}
\label{thm_MoebiusGroup}
The set of M�bius transformations forms a group under the action of function composition and can be identified with the \emph{projective general linear group} $\PGL{\C}$.
\end{theorem}
\begin{remark}
\index{GL2C@$\GL{\C}$}
\index{SL2C@$\SL{\C}$}
\index{PGL2C@$\PGL{\C}$}
\index{PSL2C@$\PSL{\C}$}
\index{General linear group}
\index{Special linear group}
\index{Projective!general linear group}
\index{Projective!special linear group}
The group $\PGL{\C}$ is obtained from the \emph{general linear group} $\GL{\C}$, \ie the set of invertible 2-by-2 matrices, by identifying all matrices which differ by a scalar multiple. Since over $\C$ every invertible matrix can be scaled such that its determinant gets 1, we also could have taken $\SL{\C}$, the \emph{special linear group} consisting of all 2-by-2 matrices with determinant 1, as basic set for this construction. This again yields the same group but this time it is denoted $\PSL{\C}$, the \emph{projective special linear group}.
\end{remark}
\begin{proof}[Proof (Theorem \ref{thm_MoebiusGroup}):]
Let $\phi$ and $\psi$ be M�bius transformations with
\begin{equation*}
\phi(z) = \moebius{a}{b}{c}{d}{z},\quad \psi(z) = \moebius{e}{f}{g}{h}{z}.
\end{equation*}

First, we make the trivial observation, that composing those two transformations again yields a rational function of the desired form:
\begin{equation}
\label{eqn_MoebiusComposition}
\phi \circ \psi(z) 
 = \moebius{a}{b}{c}{d}{\moebius{e}{f}{g}{h}{z}} 
 = \frac{aez + af + bgz + bh}{cez + cf + dgz + dh} 
 = \moebius{(ae + bg)}{(af + bh)}{(ce + dg)}{(cf + dh)}{z}
\end{equation}

Having a closer look on the resulting coefficients one might notice that they relate to the following matrix product:
\begin{equation}
\label{eqn_MatrixProduct}
\mat{a}{b}{c}{d} \cdot \mat{e}{f}{g}{h} 
 = \mat{ae + bg}{af + bh}{ce + dg}{cf + dh}
\end{equation}

This motivates the definition of a mapping $\pi$ between matrices in $\GL{\C}$ and M�bius transformations:
\begin{equation*}
\pi: \mat{a}{b}{c}{d} \mapsto \left(z \mapsto \moebius{a}{b}{c}{d}{z}\right)
\end{equation*}
Note, that the domain of $\pi$ is $\GL{\C}$, \ie the set of 2-by-2 matrices with nonzero determinant. This is perfectly consistent with the condition $ad - bc \ne 0$ we have for M�bius transformations. For this reason, $\pi$ is well-defined. 

But $\pi$ is not only a function, it is in fact a homomorphism from $\GL{\C}$ to the set of M�bius transformations, as can be seen from (\ref{eqn_MoebiusComposition}) and (\ref{eqn_MatrixProduct}). Trivially, $\pi$ is also surjective, which carries over the group structure of $\GL{\C}$ to the set of M�bius transformations. The kernel of $\pi$ comprises of all multiples of the identity matrix. Therefore, by the first isomorphism theorem the set of M�bius transformations is isomorphic to $\GL{\C}/\ker{\pi} \cong \PGL{\C}$.
\end{proof}

\subsection{Generalized circles}

\subsection{Stereographic projection}

\section{The modular group}

This section is about the modular group, a discrete subgroup of M�bius transformations.

\section{Ford circles and continued fractions}

This section is about continued fractions.
