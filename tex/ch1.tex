\chapter{Introduction}

\section{M�bius transformations}

In the following we define M�bius transformations and collect some basic properties which will be needed later on.

\begin{definition}
A non-constant rational function $f \in \C(z)$ of the form 
\begin{equation*}
f(z) = \moebius{a}{b}{c}{d}{z},\quad a, b, c, d \in \C,\quad ad - bc \ne 0
\end{equation*}
is called \defitemi{Mobius transformation}{M�bius transformation}.
\end{definition}

\begin{remark}
The condition $ad - bc \ne 0$ just ensures that $f$ is in fact non-constant.
\end{remark}

\begin{theorem}
The set of M�bius transformations forms a group under the action of function composition and can be identified with the projective general linear group $\PGL{\C}$.
\end{theorem}
\begin{proof}
Let $f, g$ be M�bius transformations with
\begin{equation*}
\phi(z) = \moebius{a}{b}{c}{d}{z},\quad \psi(z) = \moebius{e}{f}{g}{h}{z}.
\end{equation*}

First, we make the trivial observation, that composing those two transformations again yields a rational function of the desired form:
\begin{equation}
\label{eqn_MoebiusComposition}
\phi \circ \psi(z) 
 = \moebius{a}{b}{c}{d}{\moebius{e}{f}{g}{h}{z}} 
 = \frac{aez + af + bgz + bh}{cez + cf + dgz + dh} 
 = \moebius{(ae + bg)}{(af + bh)}{(ce + dg)}{(cf + dh)}{z}
\end{equation}

Having a closer look on the resulting coefficients one might notice that they relate to the following matrix product:
\begin{equation}
\label{eqn_MatrixProduct}
\mat{a}{b}{c}{d} \cdot \mat{e}{f}{g}{h} 
 = \mat{ae + bg}{af + bh}{ce + dg}{cf + dh}
\end{equation}

This motivates the definition of the function $\pi$ which assigns every invertible 2-by-2 matrix a corresponding M�bius transformation:
\begin{equation*}
\pi: \mat{a}{b}{c}{d} \mapsto \left(z \mapsto \moebius{a}{b}{c}{d}{z}\right)
\end{equation*}
Note, that the domain of $\pi$ is the $\GL{\C}$, \ie the set of 2-by-2 matrices with nonzero determinant. This is perfectly consistent with the condition $ad - bc \ne 0$ we have for M�bius transformations. For this reason, $\pi$ is well-defined. 

But $\pi$ is not only a function, it is in fact a homomorphism from the general linear group $\GL{\C}$ to the set of M�bius transformations, as can be seen from (\ref{eqn_MoebiusComposition}) and (\ref{eqn_MatrixProduct}). Trivially, $\pi$ is also surjective, which carries over the group structure of $\GL{\C}$ to the set of M�bius transformations. The kernel of $\pi$ comprises of all multiples of the identity matrix. Therefore, by the first isomorphism theorem the set of M�bius transformations is isomorphic to $\GL{\C}/\ker{\pi} \cong \PGL{\C}$, the projective general linear group.
\end{proof}

\section{The modular group}

This section is about the modular group, a discrete subgroup of M�bius transformations.

\section{Ford circles and continued fractions}

This section is about continued fractions.
