\chapter{Introduction}

\section{M�bius transformations}

In the following we define M�bius transformations and collect some basic properties which will be needed later on.

\begin{definition}
\index{Mobius transformation@M�bius transformation}
A non-constant rational function $\phi \in \C(z)$ of the form 
\begin{equation*}
\phi(z) = \moebius{a}{b}{c}{d}{z},\quad a, b, c, d \in \C,\quad ad - bc \ne 0
\end{equation*}
is called \emph{M�bius transformation}.
\end{definition}

\begin{remark}
The condition $ad - bc \ne 0$ just ensures that $\phi$ is in fact non-constant.
\end{remark}

\begin{theorem}
\label{thm_MoebiusGroup}
The set of M�bius transformations forms a group under the action of function composition and can be identified with the projective general linear group $\PGL{\C}$ or the projective special linear group $\PSL{\C}$.
\end{theorem}
\begin{remark}
\index{GL2C@$\GL{\C}$}
\index{SL2C@$\SL{\C}$}
\index{PGL2C@$\PGL{\C}$}
\index{PSL2C@$\PSL{\C}$}
\index{General linear group}
\index{Special linear group}
\index{Projective!general linear group}
\index{Projective!special linear group}
The group $\PGL{\C}$ is obtained from the \emph{general linear group} $\GL{\C}$, \ie the set of invertible 2-by-2 matrices, by identifying all matrices which differ by a scalar multiple. Since over $\C$ every invertible matrix can be scaled such that its determinant gets 1, we also could have taken $\SL{\C}$, the \emph{special linear group} consisting of all 2-by-2 matrices with determinant 1, as basic set for this construction. This again yields the same group but this time it is denoted $\PSL{\C}$, the \emph{projective special linear group}.
\end{remark}
\begin{proof}[Proof (Theorem \ref{thm_MoebiusGroup}):]
Let $\phi$ and $\psi$ be M�bius transformations with
\begin{equation*}
\phi(z) = \moebius{a}{b}{c}{d}{z},\quad \psi(z) = \moebius{e}{f}{g}{h}{z}.
\end{equation*}

First, we make the trivial observation, that composing those two transformations again yields a rational function of the desired form:
\begin{equation}
\label{eqn_MoebiusComposition}
\phi \circ \psi(z) 
 = \moebius{a}{b}{c}{d}{\moebius{e}{f}{g}{h}{z}} 
 = \frac{aez + af + bgz + bh}{cez + cf + dgz + dh} 
 = \moebius{(ae + bg)}{(af + bh)}{(ce + dg)}{(cf + dh)}{z}
\end{equation}

Having a closer look on the resulting coefficients one might notice that they relate to the following matrix product:
\begin{equation}
\label{eqn_MatrixProduct}
\mat{a}{b}{c}{d} \cdot \mat{e}{f}{g}{h} 
 = \mat{ae + bg}{af + bh}{ce + dg}{cf + dh}
\end{equation}

This motivates the definition of a mapping $\pi$ between matrices in $\GL{\C}$ and M�bius transformations:
\begin{equation*}
\pi: \mat{a}{b}{c}{d} \mapsto \left(z \mapsto \moebius{a}{b}{c}{d}{z}\right)
\end{equation*}
Note, that the domain of $\pi$ is $\GL{\C}$, \ie the set of 2-by-2 matrices with nonzero determinant. This is perfectly consistent with the condition $ad - bc \ne 0$ we have for M�bius transformations. For this reason, $\pi$ is well-defined. 

But $\pi$ is not just a function, it is in fact a homomorphism from $\GL{\C}$ to the set of M�bius transformations, as can be seen from (\ref{eqn_MoebiusComposition}) and (\ref{eqn_MatrixProduct}). Trivially, $\pi$ is also surjective, which carries over the group structure of $\GL{\C}$ to the set of M�bius transformations. The kernel of $\pi$ comprises of all multiples of the identity matrix. Therefore, by the first isomorphism theorem the set of M�bius transformations is isomorphic to $\GL{\C}/\ker{\pi} \cong \PGL{\C}$.
\end{proof}

\begin{lemma}
\label{lem_MoebiusGenerators}
The group of M�bius transformations is generated by the following basic types of transformations:

\begin{tabular}{r l l l}
\index{Translation}
\index{Rotation}
\index{Dilation}
\index{Inversion}
$\bullet$ & Translations: & $z \mapsto z + \alpha$         & $\alpha \in \C$ \\
$\bullet$ & Dilations:    & $z \mapsto \rho z$             & $\rho \in \R$ \\
$\bullet$ & Rotations:    & $z \mapsto \epo{\ii \theta} z$ & $\theta \in (-\pi,\pi]$ \\
$\bullet$ & Inversion:   & $z \mapsto \rezi{z}$           & ~ 
\end{tabular}
\end{lemma}
\begin{proof}
Let $\phi(z) = \moebius{a}{b}{c}{d}{z}$ be an arbitrary M�bius transformation. In the case when $c = 0$, we may further assume w.l.o.g. that $d = 1$ such that the transformation simply writes $\phi(z) = a z + b$. Obviously this is dilation and rotation by the factor $a$ and translation by $b$.

Let's consider the more interesting case, when $c \ne 0$. \Wlog we assume that $c = 1$, such that 
\begin{equation*}
\phi(z) = \moebius{a}{b}{}{d}{z} = a + \frac{b - ad}{z + d}.
\end{equation*}
Also in this case it is easy to see that $\phi$ is composed of translation by $d$, inversion, dilation and rotation by the factor $b - ad$ and a final translation by $a$.
\end{proof}

Now that we have defined the group of M�bius transformations, it is worth to get a better geometric intuition about how these maps act on the complex plane. Lemma \ref{lem_MoebiusGenerators} gives a first insight, as translations, dilations and rotations are quite easy to understand. Also the map $z \mapsto \rezi{z}$ has a geometric interpretation, namely as circle inversion followed by a reflection.

\index{Circle inversion}
\index{Inversion}
In 2-dimensional geometry, \emph{circle inversion} with respect to a reference circle with center $C$ and a radius $r$ takes each point $P$ on the plane to a point $P^{\prime}$ which is determined by $CP \cdot CP^{\prime} = r^2$. The image of $C$ is defined to be the point at infinity (and vice versa). Roughly speaking the inversion turns the circle ``inside out'', \ie points inside the reference circle are bijectively mapped to points outside while leaving rays from the center invariant.
% TODO #15: Figure for circle inversion

Coming back to the concrete map $z \mapsto \rezi{z}$, it can now be interpreted the following way: Circle inversion with regard to the unit circle $\UnitCirc$ maps each $z \in \C$ to $\frac{z}{\abs{z}^2} = \rezi{\conj{z}}$. Then, reflection across the real axis (\ie complex conjugation) takes $\rezi{\conj{z}}$ to $\rezi{z}$. 

Summing up, all the basic types of M�bius transformations mentioned in Lemma \ref{lem_MoebiusGenerators} have a very direct geometric interpretation. Still, arbitrary M�bius transformations (especially those involving at least one inversion) are hard to describe in a similar geometric and intuitive way. 

Luckily there is another characterization of M�bius transformations which is both, elegant and visually accessible.

% ----------------------------------------- Subsection Stereographic projection
\subsection{Stereographic projection}

This section is about the great work of Douglas Arnold and Jonathan Rogness, ``M�bius transformations revealed'' \cite{arnold2008mobius}, in which they give a characterization of M�bius transformations in terms of stereographic projections and rigid motions of spheres 3D-space.



% ---------------------------------------------- Subsection Generalized circles
\subsection{Generalized circles}

\index{Generalized circle}
From the geometric point of view M�bius transformations have the beautiful property that they preserve generalized circles. \emph{Generalized circles} are either circles (in the usual sense) or lines on the complex plane $\C$. They can also be thought of circles on the Riemann sphere $\RS$, where lines on the complex plane have a one-to-one correspondence with circles through the point $\infty$ on the Riemann sphere. In order to give an exact definition, we first make the following considerations:

A circle with center $m \in \C$ and radius $r > 0$ can be described as the set of points $z \in \C$ for which
\begin{equation*}
\abs{z - m} = r.
\end{equation*}
This is obviously equivalent to
\begin{equation*}
\abs{z - m}^2 = (z - m) \conj{(z - m)} = r^2
\end{equation*}
and
\begin{equation}
\label{eqn_Circle}
z \conj{z} - m \conj{z} - \conj{m} z + m \conj{m} - r^2 = 0.
\end{equation}
The generalization comes into play if we multiply this last equation by a constant $A \in \R$
\begin{equation*}
A z \conj{z} - A m \conj{z} - A \conj{m} z + A m \conj{m} - A r^2 = 0
\end{equation*}
and introduce constants $B$, $C$ and $D$ appropriately such that we can write it in the form
\begin{equation}
\label{eqn_GenCircle}
A z \conj{z} + B \conj{z} + C z + D = 0.
\end{equation}
Note that $D$ is real and $B = \conj{C}$ are complex conjugates. From an equation of form (\ref{eqn_GenCircle}) we can read off the center and radius of the corresponding circle by
\begin{IEEEeqnarray}{rCl}
m &=& -\frac{B}{A}, \IEEEyessubnumber \\
r &=& \sqrt{m \conj{m} - \frac{D}{A}} = \sqrt{\frac{BC - AD}{A^2}}. \IEEEyessubnumber
\end{IEEEeqnarray}
Clearly we can only do so, if $A \ne 0$ and $BC - AD > 0$. 

In the case when $A = 0$ equation (\ref{eqn_GenCircle}) can be written as
\begin{equation*}
\Re{\frac{C}{\abs{C}} z} = -\frac{D}{2 \abs{C}},
\end{equation*}
which defines a line on the complex plane. We see this by considering the simpler equation $\Re{z} = -\frac{D}{2 \abs{C}}$ first, which obviously defines a line parallel to the imaginary axis through the real point $-\frac{D}{2 \abs{C}}$. Then we observe, that the multiplication with $\frac{C}{\abs{C}}$ just rotates this line clockwise around the origin by an angle which is given by $\arg(C)$.

To conclude our considerations we finally note that equation (\ref{eqn_GenCircle}) can also be written in terms of a sesquilinear form
\begin{equation*}
\rvec{\conj{z}}{1} \cdot \mat{A}{B}{C}{D} \cdot \cvec{z}{1} = 0.
\end{equation*}
The matrix occurring in the middle is Hermitian and has a negative determinant, because of the condition $BC - AD > 0$ from above. We can now give an exact definition for generalized circles.

\begin{definition}
Let $M \in \Mat{\C}{2}{2}$ be a Hermitian matrix with $\det(M) < 0$. A \emph{generalized circle} is the set of solutions $z \in \C$ to
\begin{equation}
\label{eqn_GenCircleMat}
\rvec{\conj{z}}{1} \cdot M \cdot \cvec{z}{1} = 0.
\end{equation}
\end{definition}

\begin{remark}
Since danger of confusion is minimal, in the following we will use the same name for a generalized circle and its corresponding Hermitian matrix (which is uniquely determined up to a real scalar factor).
\end{remark}

\begin{theorem}
\label{thm_MoebiusGenCircle}
Let $M \in \Mat{\C}{2}{2}$ be a Hermitian matrix with $\det(M) < 0$ and $P \in \PGL{\C}$. The image of the generalized circle $M$ under the M�bius transformation $\phi$ corresponding to the matrix $P \in \PGL{\C}$ is the generalized circle  $\htransp{(\inv{P})} \cdot M \cdot \inv{P}$.
\end{theorem}
\begin{proof}
Let us write
\begin{equation*}
P = \mat{a}{b}{c}{d},
\end{equation*}
such that the corresponding M�bius transformation $\phi$ has the form
\begin{equation*}
\phi(z) = \moebius{a}{b}{c}{d}{z}.
\end{equation*}
First we observe the trivial fact that 
\begin{equation}
\label{eqn_HomogenousTransform}
P \cdot \cvec{z}{1} = \cvec{a z + b}{c z + d}.
\end{equation}
We need to show that for all $z$ satisfying (\ref{eqn_GenCircleMat}), $\phi(z)$ lies on the generalized circle $\htransp{(\inv{P})} \cdot M \cdot \inv{P}$, that is 
\begin{equation*}
\rvec{\conj{\phi(z)}}{1} \cdot \htransp{(\inv{P})} \cdot M \cdot \inv{P} \cdot \cvec{\phi(z)}{1} = 0.
\end{equation*}
By multiplying with the scalar $c z + d$ and its complex conjugate we obtain
\begin{equation*}
\rvec{\conj{az + b}}{\conj{cz + d}} \cdot \htransp{(\inv{P})} \cdot M \cdot \inv{P} \cdot \cvec{az + b}{cz + d} = 0.
\end{equation*}
Now we apply (\ref{eqn_HomogenousTransform}) and its Hermit transposed companion
\begin{equation*}
\rvec{\conj{z}}{1} \cdot \htransp{P} \htransp{(\inv{P})} \cdot M \cdot \inv{P} \cdot P \cdot \cvec{z}{1} = 0,
\end{equation*}
which yields after canceling out $P$ and $\inv{P}$ the desired result (\ref{eqn_GenCircleMat}).
\end{proof}


\section{The modular group}

This section is about the modular group, a discrete subgroup of M�bius transformations.

\section{Ford circles and continued fractions}

This section is about continued fractions.
