% --------------------------------------------------------------------- Symbols
\renewcommand{\phi}{\varphi}
\newcommand{\Cayley}{\Psi}
\newcommand{\ModCayley}{\Phi}
\newcommand{\ex}{e}
\newcommand{\ii}{i}
\newcommand{\col}{\mathcal{C}}

% ------------------------------------------------------------------------ Sets
\newcommand{\N}{\mathbb{N}}
\newcommand{\R}{\mathbb{R}}
\newcommand{\Q}{\mathbb{Q}}
\newcommand{\Irrat}{\mathbb{I}}
\newcommand{\Rplus}{\R^{+}}
\newcommand{\C}{\mathbb{C}}
\newcommand{\Z}{\mathbb{Z}}
\newcommand{\EC}{\C_{\infty}}
\newcommand{\EQ}{\Q_{\infty}}
%\newcommand{\RS}{\mathcal{S}}
\newcommand{\UnitCirc}{S^{1}}
\newcommand{\UnitSphere}{S_1}
\newcommand{\EU}{\mathcal{H}^\ast}

\newcommand{\Syms}{\Sigma}
\newcommand{\Words}[1]{{#1}^{\star}}
\newcommand{\FunDom}{\mathcal{F}}
\newcommand{\FunSet}{\mathcal{F}^\ast}
\newcommand{\Indisk}{\mathcal{I}}
\newcommand{\Forddisk}{\mathcal{L}}

\newcommand{\presentation}[2]{\left\langle #1 \mid #2 \right\rangle}
\newcommand{\FreeGrp}[1]{{#1}_{\sim}}

\newcommand{\LinGrp}[3]{\operatorname{#1}_{#2}(#3)}

\newcommand{\GLn}[2]{\LinGrp{GL}{#1}{#2}}
\newcommand{\SLn}[2]{\LinGrp{SL}{#1}{#2}}
\newcommand{\PGLn}[2]{\LinGrp{PGL}{#1}{#2}}
\newcommand{\PSLn}[2]{\LinGrp{PSL}{#1}{#2}}

\newcommand{\GL}[1]{\GLn{2}{#1}}
\newcommand{\SL}[1]{\SLn{2}{#1}}
\newcommand{\PGL}[1]{\PGLn{2}{#1}}
\newcommand{\PSL}[1]{\PSLn{2}{#1}}

\newcommand{\Stabilizer}[2]{{#1}_{#2}}

\newcommand{\Center}[1]{\operatorname{Z}(#1)}

\newcommand{\Mat}[3]{{#1}^{#2 \times #3}}
\newcommand{\SqMat}[2]{\Mat{#1}{#2}{#2}}

%\newcommand{\ModGrp}{\Gamma}

\newcommand{\setdefsz}[3]{#1\{ #2 \ #1| \ #3 #1\}}
\newcommand{\setdef}[2]{\setdefsz{}{#1}{#2}}

\newcommand{\fundef}[5]{#1 : \left\{
\begin{matrix} 
#2 &\to& #3 \\ 
#4 &\mapsto& #5
\end{matrix}
\right.}

% ------------------------------------------------------------------ References
\newcommand{\Poincare}{Poincar{\'e}}

\newcommand{\Caratheodory}{Carath{\'e}odory~\cite{caratheodory1961funktionentheorie}}
\newcommand{\Fenchel}{Fenchel~\cite{fenchel1989elementary}}
\newcommand{\Ford}{Ford~\cite{ford1938fractions}}
\newcommand{\Hungerford}{Hungerford~\cite{hungerford1974algebra}}
\newcommand{\Klein}{Klein/Fricke~\cite{klein1966vorlesungen}}
\newcommand{\Lehner}{Lehner~\cite{lehner1982discontinuous}}
\newcommand{\Mumford}{Mumford~\cite{mumford2002indra}}
\newcommand{\ConcreteTet}{Kauers/Paule~\cite{kauers2011concrete}}
\newcommand{\Perron}{Perron~\cite{perron1913lehre}}
\newcommand{\Petersson}{Petersson~\cite{petersson1932entwicklung}}
\newcommand{\Rademacher}{Rademacher~\cite{rademacher1939fourier}}
\newcommand{\Schoeneberg}{Schoeneberg~\cite{schoeneberg1974elliptic}}
\newcommand{\Schwerdtfeger}{Schwerdtfeger~\cite{schwerdtfeger2012geometry}}
\newcommand{\Zuellig}{Z�llig~\cite{zuellig1928geometrische}}

% ---------------------------------------------------------------------- Macros
\newcommand{\todo}[2]{\bigskip\noindent\framebox[\textwidth]{\emph{TODO #1:} #2}\bigskip}

% --------------------------------------------------------------- Abbreviations
\newcommand{\ie}{i.e.\ }
\newcommand{\eg}{e.g.\ }
\newcommand{\resp}{resp.\ }
\newcommand{\Wlog}{W.l.o.g.\ }

% ------------------------------------------------------------------- Functions
\newcommand{\lxor}{\overset{.}{\lor}}

\newcommand{\floor}[1]{\left\lfloor #1 \right\rfloor}
\newcommand{\ceil}[1]{\left\lceil #1 \right\rceil}
\newcommand{\nint}[1]{\operatorname{nint}\left( #1 \right)}
\newcommand{\sgn}[1]{\operatorname{sgn}\left( #1 \right)}

\newcommand{\topcl}[1]{\operatorname{cl}(#1)}

\newcommand{\half}[1]{\frac{#1}{2}}
\newcommand{\reci}[1]{\frac{1}{#1}}

\newcommand{\cfr}[2]{
\begin{array}{c}\multicolumn{1}{c|}{#1}\\
\hline\multicolumn{1}{|c}{#2}\end{array}}

\newcommand{\inv}[1]{{#1}^{-1}}

\newcommand{\moebius}[5]{\frac{#1 #5 + #2}{#3 #5 + #4}}
\newcommand{\crossrat}[4]{\frac{({#1}-{#2})({#3}-{#4})}{({#1}-{#3})({#2}-{#4})}}
\newcommand{\mat}[4]{\begin{pmatrix}#1 & #2 \\ #3 & #4\end{pmatrix}}
\newcommand{\smallmat}[4]{\left({}^{#1}_{#3}\ {}^{#2}_{#4}\right)}
\newcommand{\rvec}[2]{\begin{pmatrix}#1 & #2\end{pmatrix}}
\newcommand{\cvec}[2]{\begin{pmatrix}#1 \\ #2\end{pmatrix}}
\newcommand{\transp}[1]{#1^{\textrm{T}}}
\newcommand{\htransp}[1]{#1^{\textrm{H}}}
\newcommand{\id}[1]{\operatorname{id}_{#1}}

\newcommand{\abs}[1]{\left|#1\right|}
\newcommand{\conj}[1]{\overline{#1}}
\renewcommand{\Re}[1]{\operatorname{Re}\left(#1\right)}
\renewcommand{\Im}[1]{\operatorname{Im}\left(#1\right)}

\newcommand{\epo}[1]{e^{#1}}

\newcommand{\eucnorm}[1]{\left\| #1 \right\|_2}
\newcommand{\hypdist}[2]{d_\text{hyp}\left({#1},{#2}\right)}

% ---------------------------------------------------------------- Environments

\newtheorem{theorem}{Theorem}[chapter]
\newtheorem{corollary}[theorem]{Corollary}
\newtheorem{lemma}[theorem]{Lemma}

\theoremstyle{definition}
\newtheorem{definition}[theorem]{Definition}
\newtheorem{example}[theorem]{Example}

\theoremstyle{remark}
\newtheorem{remark}[theorem]{Remark}
