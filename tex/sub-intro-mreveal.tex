\subsection{Stereographic projection}

This section is about the great work of Douglas Arnold and Jonathan Rogness, ``M�bius transformations revealed'' \cite{arnold2008mobius}, in which the authors give a characterization of M�bius transformations in terms of stereographic projections and rigid motions of spheres 3D-space.

In order to introduce stereographic projection, we first have to embed $\C$ into $\R^3$. We do by using the map $\iota: \C \to \R^3,\ z \mapsto \left(\Re{z}, \Im{z}, 0\right)$, which means that we identify the complex plane with the plane $x_3 = 0$. Next we have to define admissible spheres:

\begin{definition}
A \emph{sphere} with center $c \in \R^3$ and radius $r > 0$ is the set $S := \setdef{x \in \R^3}{\eucnorm{x - c} = r}$. Its \emph{north-pole} is the unique point $n \in S$ with maximal $x_3$-coordinate. A sphere whose north pole lies in the upper half-space $H := \setdef{x \in \R^3}{x_3 > 0}$ is called an \emph{admissible sphere}.
\end{definition}

\begin{definition}
\label{dfn_StereoProject}
Let $S$ be an admissible sphere and $n \in S$ be its north pole. The \emph{stereographic projection} $P_S: \RS \to S$ assigns each $z \in \RS$ a corresponding point $x \in S$ the following way: If $z \ne \infty$, then the  unique line joining $\iota(z)$ with $n$ intersects $S$ in exactly one other point $x \ne n$, which is defined to be the image of $z$, $P_S(z) := x$ and $P_S(\infty) := n$. 
\end{definition}

Note, that the stereographic projection maps $\RS$ bijectively to every admissible sphere $S \subseteq \R^3$.

\begin{theorem}
\label{thm_MoebiusRevealed}
A function $\phi : \RS \to \RS$ is a M�bius transformation, if an only if it can be obtained by stereographic projection of $\RS$ to an admissible sphere $S \subseteq \R^3$, followed by a \emph{rigid motion}\footnote{Rigid motions are affine transformations of $\R^3$ which are only composed of rotations and translations.} T of $\R^3$ which maps $S$ to another admissible sphere $TS$, followed by inverse stereographic projection from $TS$ back to $\RS$:
\begin{equation}
\label{eqn_MoebiusRevealedForm}
\phi = P_{TS}^{-1} \circ T \circ P_S.
\end{equation}
\end{theorem}
\begin{proof}[Sketch of proof]
We will only show that each M�bius transformation $\phi$ can be realized in the form $\ref{eqn_MoebiusRevealedForm}$. The fact that all maps of this form are indeed M�bius transformations can be seen either by direct calculation or by the observation that $\phi$ corresponds to map $P_S \circ \phi \circ P_S^{-1} = P_S \circ P_{TS}^{-1} \circ T$ from $S$ to itself which is a holomorphic automorphism and applying the theorem which states that the group of holomorphic automorphisms of the Riemann sphere is given exactly by all M�bius transformations.

First we consider the four basic types of transformaions and how they can be realized in the form (\ref{eqn_MoebiusRevealedForm}):
\begin{description}
\item[Translation:] The map $z \mapsto z + \alpha,\ \alpha \in \C$ can be realized by choosing an arbitrary admissible sphere $S$ and setting $T = T_{\alpha}: x \mapsto x + \iota(\alpha)$, which simply translates $S$ in a direction parallel to the $x_3 = 0$ plane by $\iota(\alpha)$.
\item[Dilation:] The map $z \mapsto \rho z,\ \rho \in \R$ can be obtained by choosing an arbitrary admissible sphere with north pole $n$ and setting $T = D_{\rho}: x \mapsto x + \left(0, 0, (\rho-1) n_3\right)$, which moves $S$ up- ($\rho > 1$)  or downwards ($\rho < 1$) in $x_3$ direction.
\item[Rotation:] The map $z \mapsto \epo{\ii \theta} z,\ \theta \in (-\pi, \pi]$ can be realized by choosing an admissible sphere with center $c = \left(0, 0, c_3\right)$ with arbitrary $c_3$ and setting 
\begin{equation*}
T = R_{\theta}: x \mapsto 
\begin{pmatrix}
\cos{\theta} & -\sin{\theta}            & 0\\
\sin{\theta} & \phantom{+}\cos{\theta}  & 0\\
0            & \phantom{+}0             & 1
\end{pmatrix}
\cdot
\begin{pmatrix}x_1 \\ x_2 \\ x_3 \end{pmatrix},
\end{equation*}
which rotates $S$ around the $x_3$ axis by an angle of $\theta$.
\item[Inversion:] The map $z \mapsto \reci{z}$ can be realized by choosing $S$ as the unit sphere centered at the origin and setting
\begin{equation*}
T = V: x \mapsto
\begin{pmatrix}
1 & \phantom{+}0 & \phantom{+}0 \\
0 & -1           & \phantom{+}0 \\
0 & \phantom{+}0 & -1
\end{pmatrix}
\cdot
\begin{pmatrix}x_1 \\ x_2 \\ x_3 \end{pmatrix},
\end{equation*}
which is a rotation of $S$ around the $x_1$ axis by an angle of 180 degrees.
\end{description}
Now, let $\phi(z) = \moebius{a}{b}{c}{d}{z}$ be an arbitrary M�bius transformation. Following the proof of Lemma~\ref{lem_MoebiusGenerators}, we again distinguish the cases $c = 0$ and $c \ne 0$. If $c = 0$ we may assume without restriction that $d = 1$ and therefore $\phi(z) = az + b$. Clearly this map can be realized in the form (\ref{eqn_MoebiusRevealedForm}) by starting with an arbitrary admissible sphere $S$ and composing $T$ out of rotation by $\arg(a)$, dilation by $\abs{a}$ (in either order), followed by translation by $b$:
\begin{equation*}
T := T_b \circ D_{\abs{a}} \circ R_{\arg(a)}.
\end{equation*}
If $c \ne 0$, we can scale the coefficients such that $c = 1$ and write $\phi$ in the slightly more complicated form
\begin{equation*}
\phi(z) = a + \frac{b - ad}{z + d}.
\end{equation*}
In this case, we choose $S$ to be the sphere with unit radius centered at the point $\iota(-d)$ and we can again compose $T$ out of rigid motions to obtain a presentation of $\phi$ in the form (\ref{eqn_MoebiusRevealedForm}): First, translation by $d$ transforms $S$ to the unit sphere centered at the origin. Thus we can indeed apply the second transformation $V$ in order to perform an inversion. Finally we apply rotation and dilation by the factor $b - ad$ followed by a translation by $a$:
\begin{equation*}
T := T_a \circ D_{\abs{b - ad}} \circ R_{\arg(b - ad)} \circ V \circ T_d. \qedhere
\end{equation*}
\end{proof}
