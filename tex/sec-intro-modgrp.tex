\section{The modular group}

Throughout this section and also later, we adopt the notation of Schoeneberg \cite{schoeneberg1974elliptic}.

\begin{definition}
\index{Modular!group}
\index{Modular!transformation}
\index{Inhomogeneous!modular transformation}
A M�bius transformation $\phi$ of the form
\begin{equation*}
\phi(z) = \moebius{a}{b}{c}{d}{z},\quad a,b,c,d \in \Z,\quad ad - bc = 1
\end{equation*}
is called \emph{(inhomogeneous) modular transformation}.
\end{definition}

\begin{theorem}
\label{thm_ModularGroup}
\index{Modular!group}
\index{Special linear group}
\index{Projective!special linear group}
\index{PSL2Z@$\PSL{\Z}$}
\index{SL2Z@$\SL{\Z}$}
The set of modular transformations forms a discrete subgroup of the group of M�bius transformations and can be identified with the projective special linear group $\PSL{\Z}$. This group is called the \emph{modular group} and is denoted by $\ModGrp$.
\end{theorem}
\begin{proof}
The proof is very similar to that of Theorem \ref{thm_MoebiusGroup}. The only thing which has to be changed is the homomorphism $\pi$ defined in (\ref{eqn_homPi}). Its domain now is $\SL{\Z}$, the group of 2-by-2 matrices over $\Z$ with determinant 1, rather than $\GL{\C}$ (or $\SL{\C}$). Again it follows by the first isomorphism theorem, that the modular group $\ModGrp$ is isomorphic to $\SL{\Z} / \ker(\pi) \cong \PSL{\Z}$. The fact that $\ModGrp$ is a discrete subgroup of the group of M�bius transformations is now also directly evident.
\end{proof}

\begin{remark}
\index{Inhomogeneous!modular transformation}
\index{Homogeneous!modular transformation}
\index{Modular!transformation}
Sometimes the elements of $\SL{\Z}$\footnote{Schoeneberg \cite{schoeneberg1974elliptic} uses the notation $\Gamma$ for $\SL{\Z}$.} are called \emph{homogeneous modular transformations}, whereas the transformations of $\ModGrp$ are called \emph{inhomogeneous}. We will use this distinction only if it is beneficial for clarity and otherwise will just use the term ``modular transformation'' in both cases.
\end{remark}

\todo{16}{Basic mapping properties}

\begin{theorem}
The modular group is generated by the elements $U: z \mapsto z+1$ and $T: z \mapsto -\reci{z}$. 
\end{theorem}
\todo{17}{Proof: Algorithm for finding a group word in $U$ and $T$}
