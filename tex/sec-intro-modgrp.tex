\section{The modular group}

Throughout this section and also later, we adopt the notation of Schoeneberg \cite{schoeneberg1974elliptic}.

\begin{definition}
\label{dfn_ModularGroup}
\index{Modular!group}
\index{Modular!transformation}
\index{Inhomogeneous!modular transformation}
A M�bius transformation $\phi$ of the form
\begin{equation*}
\phi(z) = \moebius{a}{b}{c}{d}{z},\quad a,b,c,d \in \Z,\quad ad - bc = 1
\end{equation*}
is called \emph{(inhomogeneous) modular transformation}.
\end{definition}

\begin{theorem}
\label{thm_ModularGroup}
\index{Modular!group}
\index{Special linear group}
\index{Projective!special linear group}
\index{PSL2Z@$\PSL{\Z}$}
\index{SL2Z@$\SL{\Z}$}
The set of modular transformations forms a discrete subgroup of the group of M�bius transformations and can be identified with the projective special linear group $\PSL{\Z}$. This group is called the \emph{modular group} and is denoted by $\ModGrp$.
\end{theorem}
\begin{proof}
The proof is very similar to that of Theorem \ref{thm_MoebiusGroup}. The only thing which has to be changed is the homomorphism $\pi$ defined in (\ref{eqn_homPi}). Its domain now is $\SL{\Z}$, the group of 2-by-2 matrices over $\Z$ with determinant 1, rather than $\GL{\C}$ (or $\SL{\C}$). Again it follows by the first isomorphism theorem, that the modular group $\ModGrp$ is isomorphic to $\SL{\Z} / \ker(\pi) \cong \PSL{\Z}$. The fact that $\ModGrp$ is a discrete subgroup of the group of M�bius transformations is now also directly evident.
\end{proof}

\begin{remark}
\index{Inhomogeneous!modular transformation}
\index{Homogeneous!modular transformation}
\index{Modular!transformation}
Sometimes the elements of $\SL{\Z}$\footnote{Schoeneberg \cite{schoeneberg1974elliptic} uses the notation $\Gamma$ for $\SL{\Z}$.} are called \emph{homogeneous modular transformations}, whereas the transformations of $\ModGrp$ are called \emph{inhomogeneous}. We will use this distinction only if it is beneficial for clarity.
\end{remark}

\todo{16}{Basic mapping properties}

\begin{theorem}
\label{thm_ModGrpGen}
The modular group is generated by the elements $U: z \mapsto z+1$ and $T: z \mapsto -\reci{z}$. 
\end{theorem}
\begin{proof}
Let $A: z \mapsto \moebius{a}{b}{c}{d}{z}$ be an arbitrary modular transformation. Our goal is to show that $A$ can be written as product of the transformations $U$ and $T$. For this purpose it is more convenient to view these transformations as elements of $\PSL{\Z}$, namely
\begin{equation*}
A = \mat{a}{b}{c}{d}, \quad U = \mat{1}{1}{0}{1}, \quad T = \mat{0}{-1}{1}{\phantom{+}0}.
\end{equation*}
Because of $ad - bc = 1$, $a$ and $c$ are coprime and the Euclidean algorithm therefore yields
\begin{eqnarray*}
      a &=& q_0 \cdot c\phantom{_0} + r_1 \\
      c &=& q_1 \cdot r_1 + r_2 \\
    r_1 &=& q_2 \cdot r_2 + r_3 \\
        &\vdots& \\
r_{n-1} &=& q_n \cdot r_n + r_{n+1}\\
        &=& q_n \cdot 1\phantom{_n} + 0.
\end{eqnarray*}
We can use this to reduce the Matrix $A$ by successively multiplying powers of $U$ and $T$ from the left. Just note, that multiplication with $U^k$ adds $k$ times the second row to the first row and $T$ swaps the rows and changes the sign of one arbitrary row\footnote{This freedom of choice is due to the fact that the matrices $M$ and $-M$ are the same in $\PSL{\Z}$.}. If we concentrate only on the first column of $A$ and apply the first few transformations
\begin{equation*}
\cvec{a}{c}                           \overset{U^{-q_0}}{\longmapsto}
\cvec{r_1}{c\phantom{_1}}             \overset{T}{\mapsto} 
\cvec{\phantom{+}c\phantom{_1}}{-r_1} \overset{U^{q_1}}{\longmapsto}
\cvec{\phantom{+}r_2}{-r_1}           \overset{T}{\mapsto} 
\cvec{r_1}{r_2}                       \overset{U^{-q_2}}{\longmapsto}
\cvec{r_3}{r_2}                       \overset{T}{\mapsto}
\cvec{\phantom{+}r_2}{-r_3}           \mapsto \dots,
\end{equation*}
we soon recognize the general mapping rule, which is
\begin{IEEEeqnarray*}{rCll}
\cvec{\phantom{+}r_{j-1}}{\phantom{+}r_{j\phantom{+0}}} 
& \overset{TU^{-q_j}}{\longmapsto}
& \cvec{\phantom{+}r_{j\phantom{+0}}}{-r_{j+1}}
& \quad \text{for even } j \text{ and}\\
\cvec{\phantom{+}r_{j-1}}{-r_{j\phantom{+0}}}
& \overset{TU^{q_j}}{\longmapsto}
& \cvec{\phantom{+}r_{j\phantom{+0}}}{\phantom{+}r_{j+1}} 
& \quad \text{for odd } j.
\end{IEEEeqnarray*}
When we set $r_{-1} := a$ and $r_0 := c$, this rule is true for $0 \le j \le n$. Obviously the described procedure ends with
\begin{equation*}
\cdots \overset{T}{\mapsto}
\cvec{\phantom{+}r_{n\phantom{+0}}}{\pm r_{n+1}} = \cvec{1}{0}.
\end{equation*} 
Because we know the first column and the determinant -- which is 1 -- of the resulting matrix, we can conclude that for some $k \in \Z$ it must have the form
\begin{equation*}
\mat{1}{k}{0}{1} = U^k.
\end{equation*}
By setting $e_n := (-1)^{n} q_n$, we finally have 
\begin{equation*}
TU^{-e_n} TU^{-e_{n-1}} \cdots TU^{-e_1}TU^{-e_0} A = U^k
\end{equation*}
or equivalently,
\begin{equation*}
A = U^{e_0} TU^{e_1} \cdots TU^{e_{n-1}} TU^{e_n} T U^k,
\end{equation*}
which gives the desired representation of $A$ in terms of $U$ and $T$.
\end{proof}

It is worth formulating the algorithm used in the previous proof explicitly in the following
\begin{corollary}
\label{cor_ModGrpTUAlg}
An arbitrary modular transformation $A: z \mapsto \moebius{a}{b}{c}{d}{z}$ can be represented in terms of the transformations $U: z \mapsto z+1$ and $T : z \mapsto -\reci{z}$, by performing the following steps:
\begin{enumerate}
\item Apply the Euclidean algorithm to $a$ and $c$, with the first division being $a = q_0 \cdot c + r_1$ ($q_0$ may be 0) and let $n$ be the number of the  last division (start counting from 0). Call the arising quotients $q_0,q_1,\dots,q_n$.
\item For $j \in \{0,1,\dots,n\}$ set $e_j := (-1)^j q_j$.
\item Calculate the matrix product $TU^{-e_n} TU^{-e_{n-1}} \cdots TU^{-e_1}TU^{-e_0} A$ and multiply by $\pm 1$ in order to obtain a representation with positive diagonal elements. Read off the right-upper entry and call it $k$.
\end{enumerate}
The transformation $A$ can now be written as
\begin{equation}
\label{eqn_ModGrpTUAlg}
A = U^{e_0} TU^{e_1} \cdots TU^{e_{n-1}} TU^{e_n} T U^k.
\end{equation}
\end{corollary}

We have seen that $U$ and $T$ generate the modular group, but in order to characterize its structure, the generators $T$ and $R = TU$ better suited, as we see in the next

\begin{theorem}
\label{thm_ModGrpRel}
The generators $T: z \mapsto -\reci{z}$ and $R: z \mapsto -\reci{z+1}$ of the modular group satisfy the relations
\begin{equation}
\label{eqn_ModGrpTRRel}
T^2 = \id{\C}, \quad R^3 = \id{\C}
\end{equation}
and all other relations are derived from these two. Therefore the modular group is isomorphic to the free product of a cyclic group of order 2 and a cyclic group of order 3.
\end{theorem}

\todo{18}{Proof}
