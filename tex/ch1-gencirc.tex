\subsection{Generalized circles}

\index{Generalized circle}
From the geometric point of view M�bius transformations have the beautiful property that they preserve generalized circles. \emph{Generalized circles} are either circles (in the usual sense) or lines on the complex plane $\C$. They can also be thought of circles on the Riemann sphere $\RS$, where lines on the complex plane have a one-to-one correspondence with circles through the point $\infty$ on the Riemann sphere. In order to give an exact definition, we first make the following considerations:

A circle with center $m \in \C$ and radius $r > 0$ can be described as the set of points $z \in \C$ for which
\begin{equation*}
\abs{z - m} = r.
\end{equation*}
This is obviously equivalent to
\begin{equation*}
\abs{z - m}^2 = (z - m) \conj{(z - m)} = r^2
\end{equation*}
and
\begin{equation}
\label{eqn_Circle}
z \conj{z} - m \conj{z} - \conj{m} z + m \conj{m} - r^2 = 0.
\end{equation}
The generalization comes into play if we multiply this last equation by a constant $A \in \R$
\begin{equation*}
A z \conj{z} - A m \conj{z} - A \conj{m} z + A m \conj{m} - A r^2 = 0
\end{equation*}
and introduce constants $B$, $C$ and $D$ appropriately such that we can write it in the form
\begin{equation}
\label{eqn_GenCircle}
A z \conj{z} + B \conj{z} + C z + D = 0.
\end{equation}
Note that $D$ is real and $B = \conj{C}$ are complex conjugates. From an equation of form (\ref{eqn_GenCircle}) we can read off the center and radius of the corresponding circle by
\begin{IEEEeqnarray}{rCl}
m &=& -\frac{B}{A}, \IEEEyessubnumber \\
r &=& \sqrt{m \conj{m} - \frac{D}{A}} = \sqrt{\frac{BC - AD}{A^2}}. \IEEEyessubnumber
\end{IEEEeqnarray}
Clearly we can only do so, if $A \ne 0$ and $BC - AD > 0$. 

In the case when $A = 0$ equation (\ref{eqn_GenCircle}) can be written as
\begin{equation*}
\Re{\frac{C}{\abs{C}} z} = -\frac{D}{2 \abs{C}},
\end{equation*}
which defines a line on the complex plane. We see this by considering the simpler equation $\Re{z} = -\frac{D}{2 \abs{C}}$ first, which obviously defines a line parallel to the imaginary axis through the real point $-\frac{D}{2 \abs{C}}$. Then we observe, that the multiplication with $\frac{C}{\abs{C}}$ just rotates this line clockwise around the origin by an angle which is given by $\arg(C)$.

To conclude our considerations we finally note that equation (\ref{eqn_GenCircle}) can also be written in terms of a sesquilinear form
\begin{equation*}
\rvec{\conj{z}}{1} \cdot \mat{A}{B}{C}{D} \cdot \cvec{z}{1} = 0.
\end{equation*}
The matrix occurring in the middle is Hermitian and has a negative determinant, because of the condition $BC - AD > 0$ from above. We can now give an exact definition for generalized circles.

\begin{definition}
Let $M \in \Mat{\C}{2}{2}$ be a Hermitian matrix with $\det(M) < 0$. A \emph{generalized circle} is the set of solutions $z \in \C$ to
\begin{equation}
\label{eqn_GenCircleMat}
\rvec{\conj{z}}{1} \cdot M \cdot \cvec{z}{1} = 0.
\end{equation}
\end{definition}

\begin{remark}
Since danger of confusion is minimal, in the following we will use the same name for a generalized circle and its corresponding Hermitian matrix (which is uniquely determined up to a real scalar factor).
\end{remark}

\begin{theorem}
\label{thm_MoebiusGenCircle}
Let $M \in \Mat{\C}{2}{2}$ be a Hermitian matrix with $\det(M) < 0$ and $P \in \PGL{\C}$. The image of the generalized circle $M$ under the M�bius transformation $\phi$ corresponding to the matrix $P \in \PGL{\C}$ is the generalized circle  $\htransp{(\inv{P})} \cdot M \cdot \inv{P}$.
\end{theorem}
\begin{proof}
Let us write
\begin{equation*}
P = \mat{a}{b}{c}{d},
\end{equation*}
such that the corresponding M�bius transformation $\phi$ has the form
\begin{equation*}
\phi(z) = \moebius{a}{b}{c}{d}{z}.
\end{equation*}
First we observe the trivial fact that 
\begin{equation}
\label{eqn_HomogenousTransform}
P \cdot \cvec{z}{1} = \cvec{a z + b}{c z + d}.
\end{equation}
We need to show that for all $z$ satisfying (\ref{eqn_GenCircleMat}), $\phi(z)$ lies on the generalized circle $\htransp{(\inv{P})} \cdot M \cdot \inv{P}$, that is 
\begin{equation*}
\rvec{\conj{\phi(z)}}{1} \cdot \htransp{(\inv{P})} \cdot M \cdot \inv{P} \cdot \cvec{\phi(z)}{1} = 0.
\end{equation*}
By multiplying with the scalar $c z + d$ and its complex conjugate we obtain
\begin{equation*}
\rvec{\conj{az + b}}{\conj{cz + d}} \cdot \htransp{(\inv{P})} \cdot M \cdot \inv{P} \cdot \cvec{az + b}{cz + d} = 0.
\end{equation*}
Now we apply (\ref{eqn_HomogenousTransform}) and its Hermit transposed companion
\begin{equation*}
\rvec{\conj{z}}{1} \cdot \htransp{P} \htransp{(\inv{P})} \cdot M \cdot \inv{P} \cdot P \cdot \cvec{z}{1} = 0,
\end{equation*}
which yields after canceling out $P$ and $\inv{P}$ the desired result (\ref{eqn_GenCircleMat}).
\end{proof}
