\chapter*{Kurzfassung}

Ziel dieser Diplomarbeit ist die Visualisierung einiger grundlegender Ergebnisse aus dem Umfeld der Theorie der modularen Gruppe sowie der modularen Funktionen unter Zuhilfenahme der Computer Algebra Software \emph{Mathematica}.

Die Arbeit gliedert sich in drei Teile. Im ersten Kapitel werden f�r diese Arbeit relevante Begriffe aus der Gruppentheorie zusammengefasst. Weiters werden M�bius Transformationen eingef�hrt und deren grundlegende geometrische Eigenschaften untersucht.

Das zweite Kapitel ist der Untersuchung der modularen Gruppe aus algebraischer und geometrischer Sicht gewidmet. Der kanonische Fundamentalbereich der modularen Gruppe sowie die daraus abgeleitete Kachelung der oberen Halbebene wird eingef�hrt. Weiters wird eine generelle Methode zum Auffinden von Fundamentalbereichen f�r Untergruppen der modularen Gruppe vorgestellt, welche sich zentral auf die Konzepte der 2-dimensionalen hyperbolischen Geometrie st�tzt.

Im dritten Kapitel geben wir einige konkrete Beispiele, wie die aufgebaute Theorie f�r die Visualisierung bestimmter mathematischer Sachverhalte angewendet werden kann. Neben der Visualisierung von Graphen modularer Funktionen stellt sich auch der Zusammenhang zwischen modularen Transformationen und Kettenbr�chen als besonders sch�nes Ergebnis dar.

\chapter*{Abstract}

The aim of this diploma thesis is the visualization of some fundamental results in the context of the theory of the modular group and modular functions. For this purpose the computer algebra software \emph{Mathematica} is utilized.

The thesis is structured in three parts. In Chapter 1, we summarize some important basic concepts of group theory which are relevant to this work. Moreover, we introduce M�bius transformations and study their geometric mapping properties.

Chapter 2 is devoted to the study of the modular group from an algebraic and geometric point of view. We introduce the canonical fundamental region which gives rise to the modular tessellation of the upper half-plane. Additionally, we present a general method for finding fundamental regions with respect to subgroups of the modular group based on the concepts of 2-dimensional hyperbolic geometry.

In Chapter 3 we give some concrete examples how the developed results and methods can be exploited for the visualization of certain mathematical results. Besides the visualization of function graphs of modular functions, a particularly nice result is the connection between modular transformations and continued fraction expansions.

