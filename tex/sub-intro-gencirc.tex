\subsection{Generalized circles}

\index{Generalized!circle}
From the geometric point of view, M�bius transformations have the beautiful property that they preserve generalized circles. \emph{Generalized circles} are either circles (in the usual sense) or lines on the complex plane $\C$. They can also be thought of circles on the Riemann sphere (\ie the extended complex plane $\EC$ projected to the unit sphere $\UnitSphere$, see Remark~\ref{rem_RiemannSphere}), where lines on the complex plane stand in a one-to-one correspondence to circles through the point $\infty$ on the Riemann sphere. In order to give an exact definition, we first make the following considerations:

A circle with center $m \in \C$ and radius $r > 0$ can be described as the set of points $z \in \C$ for which
\begin{equation*}
\abs{z - m} = r.
\end{equation*}
This is obviously equivalent to
\begin{equation*}
\abs{z - m}^2 = (z - m) \conj{(z - m)} = r^2
\end{equation*}
and
\begin{equation}
\label{eqn_Circle}
z \conj{z} - m \conj{z} - \conj{m} z + m \conj{m} - r^2 = 0.
\end{equation}
The generalization comes into play if we multiply this last equation by a constant $A \in \R$
\begin{equation*}
A z \conj{z} - A m \conj{z} - A \conj{m} z + A m \conj{m} - A r^2 = 0
\end{equation*}
and introduce constants $B$, $C$ and $D$ appropriately such that we can write it in the form
\begin{equation}
\label{eqn_GenCircle}
A z \conj{z} + B \conj{z} + C z + D = 0.
\end{equation}
Note that $D$ is real and $B = \conj{C}$ are complex conjugates. From an equation of form (\ref{eqn_GenCircle}) we can read off the center and radius of the corresponding circle by
\begin{IEEEeqnarray}{rCl}
m &=& -\frac{B}{A}, \IEEEyessubnumber \\
r &=& \sqrt{m \conj{m} - \frac{D}{A}} = \sqrt{\frac{BC - AD}{A^2}}. \IEEEyessubnumber
\end{IEEEeqnarray}
Clearly we can only do so, if $A \ne 0$ and $BC - AD > 0$. 

In the case when $A = 0$, equation (\ref{eqn_GenCircle}) can be written as
\begin{equation*}
\Re{\frac{C}{\abs{C}} z} = -\frac{D}{2 \abs{C}},
\end{equation*}
which defines a line on the complex plane. We see this by considering the simpler equation $\Re{z} = -\frac{D}{2 \abs{C}}$ first (we omit the factor $\frac{C}{\abs{C}}$), which obviously defines a line parallel to the imaginary axis through the real point $-\frac{D}{2 \abs{C}}$. Then we observe that the multiplication with $\frac{C}{\abs{C}}$ just rotates this line clockwise around the origin by an angle which is given by $\arg(C)$.

To conclude our considerations, we note that equation (\ref{eqn_GenCircle}) can also be written in matrix form:
\begin{equation*}
\rvec{\conj{z}}{1} \cdot \mat{A}{B}{C}{D} \cdot \cvec{z}{1} = 0.
\end{equation*}
Next, we substitute $z = u / v$, with $u,v \in \C$, $v \ne 0$ and scale by $\conj{v} \cdot v = \abs{v}^2 > 0$, which yields the equivalent form
\begin{equation}
\label{eqn_GenCircleMatForm}
\rvec{\conj{u}}{\conj{v}} \cdot \mat{A}{B}{C}{D} \cdot \cvec{u}{v} = 0.
\end{equation}
By introducing the convention to identify $\infty \in \EC$ with any formal quotient $u/0$, $u \in \C \setminus \{0\}$, the latter form makes sense for all $z = u/v \in \EC$. 

Finally, we remember that the matrix in equation (\ref{eqn_GenCircleMatForm}) has a negative determinant, because of the condition $BC - AD > 0$ from above. Moreover, it is a Hermitian matrix, a notion which we will shortly recall:
\begin{definition}
\label{dfn_HermitianMatrix}
\index{Hermitian!matrix}
\index{Hermitian!transpose}
\index{Conjugate transpose}
Let $n > 0$ and $M \in \Mat{\C}{n}{n}$. The matrix
\begin{equation*}
\htransp{M} := \transp{\overline{M}}
\end{equation*}
obtained by complex conjugation and transposition of $M$ is called \emph{Hermitian transpose} or \emph{conjugate transpose} of $M$. If $M$ has the property $\htransp{M} = M$, it is called a \emph{Hermitian} matrix.
\end{definition}
Having now the right properties and vocabulary at hand, we can give an exact definition for generalized circles.
\begin{definition}
\label{dfn_GenCircle}
Let $M \in \Mat{\C}{2}{2}$ be a Hermitian matrix with $\det(M) < 0$. A \emph{generalized circle} is the set of solutions $u/v \in \EC$, with $u,v \in \C$ -- not both zero, to
\begin{equation}
\label{eqn_GenCircleDfn}
\rvec{\conj{u}}{\conj{v}} \cdot M \cdot \cvec{u}{v} = 0.
\end{equation}
\end{definition}

Since there should be no danger of confusion, we will from now on use the same name for a generalized circle and its corresponding Hermitian matrix (which is uniquely determined up to a real scalar factor).

\begin{remark}
Consider a circle on the Riemann sphere going through its north pole (\ie the point $\infty$). The image of this circle under stereographic projection consists of a line on the complex plane $\C$ plus the formal element $\infty \in \EC$. Therefore, exactly the generalized circles which correspond to lines should contain the point $\infty$. Indeed this requirement is satisfied by Definition~\ref{dfn_GenCircle}, as equation (\ref{eqn_GenCircleDfn}) reduces to $A = 0$ for the case $u/v = 1/0 = \infty$.
\end{remark}

Replacing the equality sign `$=$' in the equation $\abs{z - m} = r$ with `$<$' or `$\le$' and repeating the considerations from above, naturally leads to the notions of generalized open disks and generalized closed disks, which we will define now.

\begin{definition}
\index{Generalized!open disk}
\index{Generalized!closed disk}
\label{dfn_GenDisk}
Let $M \in \Mat{\C}{2}{2}$ be a Hermitian matrix with $\det(M) < 0$. A \emph{generalized open disk} is the set of solutions $u/v \in \EC$, with $u,v \in \C$ -- not both zero, to
\begin{equation}
\label{eqn_GenOpenDiskDfn}
\rvec{\conj{u}}{\conj{v}} \cdot M \cdot \cvec{u}{v} < 0
\end{equation}
and a \emph{generalized closed disk} is the set of solutions $u/v \in \EC$ to
\begin{equation}
\label{eqn_GenClosedDiskDfn}
\rvec{\conj{u}}{\conj{v}} \cdot M \cdot \cvec{u}{v} \le 0.
\end{equation}
\end{definition}

\begin{remark}
In contrast to generalized circles, the defining matrix $M$ of a generalized disk is unique up to a \emph{positive} real scalar factor. Switching from $M$ to $-M$ turns the generalized disk inside out.
\end{remark}

It is now easy to show that generalized circles and generalized disks are preserved under M�bius transformations.

\begin{theorem}
\label{thm_MoebiusGenCircle}
Let $M \in \Mat{\C}{2}{2}$ be a Hermitian matrix with $\det(M) < 0$ and $P \in \GL{\C}$. The image of the generalized circle (generalized open/closed disk) $M$ under the M�bius transformation $\phi$ corresponding to the matrix $P \in \GL{\C}$ is the generalized circle (generalized open/closed disk) $\htransp{(\inv{P})} \cdot M \cdot \inv{P}$.
\end{theorem}
\begin{proof}
Let us write $P = \smallmat{a}{b}{c}{d}$, such that the corresponding M�bius transformation $\phi$ has the form
\begin{equation*}
\phi(z) = \moebius{a}{b}{c}{d}{z}.
\end{equation*}
Now we set $w = \phi(z)$ and show that 
\begin{equation*}
\rvec{\conj{z}}{1} \cdot M \cdot \cvec{z}{1} 
\ \left\{\ \begin{matrix} = 0 \\ < 0\\ \le 0 \end{matrix}\right.
\end{equation*}
if and only if
\begin{equation*}
\rvec{\conj{w}}{1} \cdot \htransp{(\inv{P})} \cdot M \cdot \inv{P} \cdot \cvec{w}{1} 
\ \left\{\ \begin{matrix} = 0 \\ < 0\\ \le 0 \end{matrix}\right.
\end{equation*}
But this follows immediately from
\begin{equation*}
P \cdot \cvec{z}{1} = \cvec{a z + b}{c z + d} = (c z + d) \cvec{w}{1}.\qedhere
\end{equation*}
\end{proof}
