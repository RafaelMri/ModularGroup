\chapter*{Preface}

The centerpiece of the present diploma thesis, \emph{Computer Algebra and Analysis: Complex Variables Visualized}, is the modular group. The modular group plays an important role in many areas of mathematics, as for example in number theory due to its connection with partition numbers or continued fractions.

This thesis is structured in three parts. In Chapter 1, we introduce basic notions and definitions which are fundamental for the rest of this work. Firstly, we recapitulate the most important basic concepts of group theory. Moreover, we introduce \emph{M�bius transformations} and study their connection to \emph{stereographic projection} in detail. Finally, we define the concept of \emph{generalized circles} and \emph{generalized disks}, using an elegant characterization in terms of Hermitian matrices which turns out to be particularly advantageous in interaction with M�bius transformations. 

Chapter 2 is devoted to the study of the modular group from an algebraic and geometric point of view. Two different and independent algorithms, the $T$-$U$ algorithm and the $T$-$R$ algorithm, are presented which both yield group word representations for arbitrary modular transformations in terms of the transformations $T : z \mapsto \reci{z}$, $U : z \mapsto z+1$ and $R : z \mapsto \reci{z+1}$. Geometric considerations come into play when introducing \emph{fundamental regions} for the action of the modular group on the extended complex plane. A canonical fundamental region is derived which gives rise to the \emph{modular tessellation of the upper-half plane}. Lastly, the basic concepts of 2-dimensional \emph{hyperbolic geometry} are introduced in order to present an alternative and more general method finding fundamental regions. This method gives rise to so-called \emph{normal polygons} and works as well for subgroups of the modular group.

Finally, Chapter 3 has a clear emphasis on visualization. Firstly, \emph{generalized matrix powers} are introduced as device for visualization of continuous transitions between given sets and their M�bius-transformed images. Secondly, the relation between the modular transformations, \emph{Ford circles} and \emph{continued fractions} is studied in detail and visually explained using the continued fraction expansion of the irrational number $\pi$ as an example. Lastly, documenting the important role of the modular group within complex analysis, the most basic results from the theory of \emph{modular functions} are summarized. Using an adequate color coding, graphs of certain selected modular functions are depicted whose inherent visual aesthetics and symmetry reflects well the beauty of this theory.

\section*{Achievements and Outlook}

Complementary to this thesis, the \emph{Mathematica} package \emph{ModularGroup} has been developed which may be downloaded from the website of the \emph{Research Institute for Symbolic Computation} (RISC).\footnote{http://www.risc.jku.at/} The package contains essentially all algorithms described in this thesis as well as functions for visualization of generalized circles, generalized disks, modular tessellations and more. The main attention has been paid to an efficient and fast implementation of these algorithms, relying in many aspects on pre-compiled functions using machine-precision integer and floating point arithmetic rather than \emph{Mathematica}'s standard arbitrary precision arithmetic. This allows \emph{interactive visualizations} on standard desktop computers. Note that all figures contained in this thesis are based on this package. 

Furthermore it is worth noting that this thesis also contains ideas which have not been found directly in this form in the referenced literature. During the implementation of the enumeration algorithm for modular transformations, it got apparent that the left- or rightmost symbol of the unique $T$-$R$ group word of any given modular transformation can be read off directly from its corresponding matrix. This observation led to the $T$-$R$ algorithm of Section~\ref{sec_ModularGroupGenRel} and to the alternative proof for the presentation for the modular group in terms of the generators $T$ and $R$ (Theorem~\ref{thm_ModGrpTRProd}). 

For the proof that the region $\FunDom := \setdef{z \in \C}{\abs{\Re{z}} < 1 \land \abs{z} > 1}$ is a fundamental region for the action of the modular group on the upper half-plane, we intentionally take a different and slightly longer path than for example \Klein{} or \Schoeneberg{}. We first derive a fundamental region for the action of the homogeneous modular group on $\C^2$ by looking for representative vectors of minimal Euclidean norm. By carrying over the result to the inhomogeneous case, we indeed obtain the fundamental region $\FunDom$ in a very natural and instructive way.

%As the Euclidean algorithm prominently appears at several different places within this thesis, we take into account that there is more than one ``Euclidean algorithm'', depending on the chosen method for integer divison.

For visualization purposes, there are also several 
