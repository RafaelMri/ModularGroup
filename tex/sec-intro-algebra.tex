\section{Groups and basic algebraic constructions}

In this section, we will recapitulate some basic algebraic concepts, most important the notions of \emph{free groups} and \emph{free products}, and the construction of a group in terms of \emph{generators} and \emph{relations}. Additionally, we give the definitions of the \emph{(projective) general} and \emph{special linear groups} which may be known from linear algebra. A reader familiar to this concepts may readily skip this section. Moreover, we will give no rigorous proofs here, as they may be found in every good algebra book, as for example in Hungerford \cite{hungerford1974algebra}.

\subsection{Free monoids}
\index{Free!monoid}
Let $\Sigma$ be a set of formal symbols, for example $\Sigma = \{a, b, c, \dots\}$. Then, the set of words over the alphabet $\Sigma$ is defined as
\begin{equation}
\label{eqn_SigmaStarDef}
\Words{\Sigma} := \bigcup_{n\ge0} \Sigma^n.
\end{equation}
We will call the tuples of $\Words{\Sigma}$ \emph{words}. To make things easier, we will omit the parentheses when notating such a word $w \in \Words{\Sigma}$, e.g. $w = (a,b,b,a) =: abba$. Note, that also the empty tuple, which we will call the \emph{empty word} and denote as $\epsilon$, is an element of $\Words{\Sigma}$. It is now just natural to define a binary operation $\cdot$ on $\Words{\Sigma}$ which is given by concatenation of words:
\begin{equation*}
w_1 \cdot w_2 = w_1 w_2, \quad w_1, w_2 \in \Words{\Sigma}.
\end{equation*}
Obviously, $\Words{\Sigma}$ is closed under this operation and $\epsilon$ is a the neutral element. Thus $\Words{\Sigma}$ together with the operation $\cdot$ forms a monoid\footnote{Note, that in the exceptional case, when $\Sigma = \emptyset$ and thus $\Words{\Sigma} = \{\epsilon\}$, we obtain the trivial monoid by this construction.}. 

\begin{definition}
\label{dfn_FreeMonoid}
Let $\Sigma$ be an arbitrary set of symbols (called the alphabet), and $\Words{\Sigma}$ be defined as in (\ref{eqn_SigmaStarDef}). Moreover, denote concatenation of words by $\cdot$ and let $\epsilon$ be the empty word. Then, the algebraic structure $\langle \Words{\Sigma}, \cdot, \epsilon \rangle$ is called the \emph{free monoid over the alphabet $\Sigma$}.
\end{definition} 

\subsection{Free groups}
\index{Free!group}
In a similar fashion, we can construct also a group from any given formal alphabet $\Sigma$. For this purpose we first choose a disjoint copy of $\Sigma$, denoted by $\inv{\Sigma}$, and a bijection $f : \Sigma \to \inv{\Sigma}$. We can extend this bijection between the sets $\Sigma$ and $\inv{\Sigma}$ to an involution on their union $\overline{\Sigma} := (\Sigma \cup \inv{\Sigma})$ by simply defining $f(f(a)) := a$ for all $a \in \Sigma$. Now we introduce the notation $f(a) =: \inv{a}$ for all $a \in \overline{\Sigma}$ and call $\inv{a}$ the \emph{(formal) inverse} of the $a$.

\index{Reduced form}
\index{Group!word}
Next, we construct the free monoid $\Words{\overline{\Sigma}}$. If $\sigma_1, \sigma_2, \dots, \sigma_n$ are symbols of $\overline{\Sigma}$, then we say the word $\sigma_1 \sigma_2 \dots \sigma_n$ is \emph{reduced}, if and only if no two subsequent symbols of the word are inverse to each other, that is
\begin{equation}
\label{eqn_GrpWordReducedForm}
 \sigma_j \ne \sigma_{j+1}^{-1} \quad \text{for all } 1 \le j < n.
\end{equation}
Clearly, every word $w \in \Words{\overline{\Sigma}}$ can be brought into reduced form by successively ``canceling out'' adjacent inverse symbols until finally a word is obtained, which satisfies (\ref{eqn_GrpWordReducedForm}). We call the result of this procedure the \emph{reduced form of $w$}. Additionally, we define two words $w_1, w_2 \in \Words{\overline{\Sigma}}$ to be \emph{equivalent}, if and only if they have the same reduced form and we write $w_1 \sim w_2$ in this case. It is easy to see that this relation is a congruence relation (\ie it is compatible with word-concatenation)\footnote{Denote the reduced form of a word $w$ as $\phi(w)$. If $w_1 \sim w_2$ and $v_1 \sim v_2$ are given, then we also have $w_1 v_1 \sim w_2 v_2$ because $\phi(w_1) = \phi(w_2) := w$ and $\phi(v_1) = \phi(v_2) := v$ implies $\phi(w_1 v_1) = \phi(w v) = \phi(w_2 v_2)$.} and thus we can consider also the set of equivalence classes $\Words{\overline{\Sigma}}/_{\sim}$ as monoid under the operation of word-concatenation. Obviously the set of reduced words is a representative system for $\Words{\overline{\Sigma}}/_{\sim}$ and we agree to denote an equivalence class $[w]_{\sim}$ simply by the reduced word $w$. $\Words{\overline{\Sigma}}/_{\sim}$ is not just a monoid, but in fact a group, as the inverse of a word $\sigma_1 \sigma_2 \dots \sigma_{n-1} \sigma_n$ is trivially given by $\sigma_n^{-1} \sigma_{n-1}^{-1} \dots \sigma_2^{-1} \sigma_1^{-1}$. We call a reduced word of $\Words{\overline{\Sigma}}$ (resp. its corresponding equivalence class in $\Words{\overline{\Sigma}}/_{\sim}$) a \emph{group word over the alphabet $\Sigma$}.

\begin{definition}
\label{dfn_FreeGroup}
Let $\Sigma$ be an arbitrary set of formal symbols and define $\overline{\Sigma} := \Sigma \cup \inv{\Sigma}$ as above. On the free monoid $\Words{\overline{\Sigma}}$ define an equivalence relation $\sim$ by identifying words with same reduced form. Then, the algebraic structure $\langle \Words{\overline{\Sigma}}/_{\sim}, \cdot, \epsilon \rangle$ is called the \emph{free group over the alphabet $\Sigma$}.
\end{definition}

\begin{remark}
In the exceptional case, when $\Sigma = \emptyset$,  we obtain the trivial group by this construction\footnote{If $\Sigma = \emptyset$ then we can also choose $\inv{\Sigma} = \emptyset$ ($\Sigma$ and $\inv{\Sigma}$ are then disjoint as required). It follows that also $\overline{\Sigma}$ is empty and we end up with the trivial monoid $\{\epsilon\}$, which is in the same time the trivial group.}.
\end{remark}

\begin{example}
Let $\Sigma$ be the singleton set $\{a\}$. Now, the free group over $\Sigma$ consists precisely of the following group words:
\begin{equation*}
\Words{\overline{\Sigma}}/_{\sim} = \{\epsilon, a, aa, aaa, \dots, \inv{a}, \inv{a}\inv{a},\dots\}.
\end{equation*}
If we now agree on the notation
\begin{equation*}
\epsilon =: a^0, \quad 
\underbrace{a a \dots a}_{k \text{ times}} =: a^k, \quad
\underbrace{\inv{a} \inv{a} \dots \inv{a}}_{k \text{ times}} =: a^{-k},
\end{equation*}
we have $a^n \cdot a^m = a^{n+m}$ for all $n,m \in \Z$ and it is thus evident, that the free group over any one-element alphabet is isomorphic to the group $\langle \Z, +, 0 \rangle$. 

Note, that this is the only case (besides the trivial one, when $\Sigma = \emptyset$), where the free group is commutative. In fact, the free group construction always yields a group with ``richest possible'' structure in the following sense:

\begin{theorem}
\label{thm_FreeGrpUniqueHom}
Let $\Sigma$ be an arbitrary set and denote the free group over $\Sigma$ as $F$. Let $G$ be an arbitrary group. If any mapping $f : \Sigma \to G$ is given, then we can always extend $f$ in a unique way to a homomorphism $\overline{f} : F \to G$.
\end{theorem}

In fact, the property described in Theorem \ref{thm_FreeGrpUniqueHom} completely characterizes the free group up to isomorphism, as stated in the next Theorem:

\begin{theorem}
\label{thm_FreeGrpUnivMapProp}
Let $\Sigma$ be an arbitrary set and $F$ be a group together with an injective map $\iota : \Sigma \to F$. If $F$ has the property, that for any map $f : \Sigma \to G$, where $G$ is an arbitrary group, the map $f \circ \inv{\iota}: F \to G$ can be extended uniquely to a homomorphism $\overline{f}: F \to G$, then $F$ is (isomorphic to) the free group over the alphabet $\Sigma$.
\end{theorem}

\end{example}

\subsection{Generators and relations}

\index{Generator}
\index{Group!generator}
\begin{definition}
Let $G$ be a group and $\Sigma \subseteq G$ a subgroup. We say $G$ is \emph{generated by} the elements of $\Sigma$, if every element $g \in G$ can be written as a product of elements from $\Sigma$, that is 
\begin{equation}
\label{eqn_GrpGenerators}
\forall g \in G\ \exists n \in \N,\ (\sigma_j) \in \Sigma^n,\ (k_j) \in \Z^n:\quad 
g = \sigma_1^{k_1} \sigma_2^{k_2} \dots \sigma_n^{k_n}.
\end{equation}
Moreover, in this case we call the elements of $\Sigma$ \emph{generators} of $G$. 
\end{definition}

Let us first consider the case, when some $g \in G$ can be generated in two  different ways, for example
\begin{equation*}
g = \sigma_1^{k_1} \sigma_2^{k_2} \dots \sigma_n^{k_n} 
  = \tau_1^{\ell_1} \tau_2^{\ell_2} \dots \tau_m^{\ell_m}
\end{equation*}
with $\sigma_j, \tau_j \in \Sigma$ and $k_j, \ell_j \in \Z$. This immediately gives (here, we denote the neutral element by $e$)
\begin{equation}
\label{eqn_GrpRelation}
\sigma_1^{k_1} \sigma_2^{k_e} \dots \sigma_n^{k_n} \cdot
\tau_m^{-\ell_m} \tau_{m-1}^{-\ell_{m-1}} \dots \tau_1^{-\ell_1} = e.
\end{equation}
\index{Group!relation}
We call the group word occurring on the left hand side of (\ref{eqn_GrpRelation}) a \emph{relation} on $G$. If we denote by $F$ the free group over the alphabet $\Sigma$, the set of all such possible relations on $G$ forms a normal subgroup $N \unlhd F$. Moreover, $G$ is isomorphic to the factor group $F/N$. In the other case, when the product representation in (\ref{eqn_GrpGenerators}) is unique for all $g \in G$, then we have just the trivial relation $e = e$ and the normal subgroup $N$ thus consists just of the neutral element. $G$ is then said to be \emph{relation-free} -- that is where the terminology ``free group'' actually comes from -- and $G \cong F$. 

Summing up, we see that every group can be described by supplying a set of generators $\Sigma$ and a the set of relations $N \unlhd \Words{\overline{\Sigma}}$. Of course, it is also sufficient to just supply a subset $R$ of relations from which the other relations can be derived. This leads to the following definition.

\begin{definition}
\label{dfn_GrpConstructGenRel}
Let $\Sigma$ be an arbitrary set and $R \subseteq \Words{\overline{\Sigma}}$ a set of group words over the alphabet $\Sigma$. A group $G$ is said to be the \emph{group defined by} the \emph{generators} $\sigma \in \Sigma$ \emph{and relations} $R$, if $G$ is obtained by the following construction:
\begin{enumerate}
\item Construct the free group $F = \Words{\overline{\Sigma}}/_{\sim}$.
\item Let $N$ be the normal subgroup of $F$ generated by the relations $R$:
\begin{equation*}
N := \bigcap_{R \subseteq N^{\prime} \unlhd F} N^{\prime}.
\end{equation*}
\item Define $G := \presentation{\Sigma}{R} := F/N$.
\end{enumerate}
Finally, $\presentation{\Sigma}{R}$ is called a \emph{presentation} of $G$.

For easier notation, we may omit braces, when enumerating $\Sigma$ and $R$ in a presentation of $G$ and moreover we may write out relations more explicitly, for example instead of $G = \presentation{\{a\}}{\{a^2\}}$ we may write $G = \presentation{a}{a^2 = 1}$.
\end{definition}

\begin{remark}
\index{Word problem}
The presentation $\presentation{\Sigma}{R}$ of a group $G$ is by far not unique as the normal group $N$ can be generated by many different subsets $R \subseteq N$. Even worse, the $\emph{word problem}$, \ie the question, if a given group word $w \in \Words{\overline{\Sigma}}/_{\sim}$ is in $N$ when $R$ is given, is in general not decidable, which means that there is no general algorithm for deciding, if two given group words represent the same element of $G$.
\end{remark}

\begin{example}
The group $\presentation{t,r}{t^2 = r^3 = 1}$ consists of all group words of the form
\begin{equation*}
\label{ex_RTGroup}
r^{k_1} t r^{k_2} t \dots t r^{k_n}, \quad 
\text {with } k_1, k_n \in \{0,\pm 1\} \text{ and } k_2, \dots, k_{n-1} \in \{\pm 1\}.
\end{equation*}
\end{example}
We conclude this section with the statement, that among all groups, which satisfy a given set of relations $R$, the group constructed as above is in a certain sense the largest possible one.

\begin{theorem}
Let $\presentation{\Sigma}{R}$ be a presentation of the group $G$. If $H$ is any group generated by $\Sigma$ and $H$ satisfies all relations $R$, then there is an unique epimorphism $G \to H$.
\end{theorem}

\subsection{Free products}

\index{Free!product}
Let $\langle G_i, \cdot, e_i \rangle_{i \in I}$ a family of pairwise disjoint groups, \ie $i \ne j \Rightarrow G_i \cap G_j = \emptyset$. We define the \emph{free product} of the family of groups $(G_i)_{i \in I}$ by a construction in terms of generators and relations (see Definition \ref{dfn_GrpConstructGenRel}) as follows. As the set of generators we choose the alphabet $\Sigma = \bigcup_{i \in I} G_i$ and as the set of relations we take all the relations coming form any of the groups $G_i$. Note, that a priori the neutral elements $e_i \in G_i$ are all different symbols, but the final factorization by the normal subgroup $N$ automatically identifies them with each other, as $N$ includes all the $e_i$ as trivial relations.

\begin{definition}
Let $(G_i)_{i \in I}$ a family of disjoint groups. For each $i \in I$, let $\pi_i : \Words{G_i} \to G_i$ the map, which evaluates words over the alphabet $G_i$ to concrete elements of $G_i$ in the obvious way. The \emph{free product} of the family $(G_i)_{i \in I}$ is defined as
\begin{equation*}
{\prod_{i \in I}}^{\star} G_i := \presentation{\Sigma}{R}, 
\quad \text{where } \Sigma = \bigcup_{i \in I} G_i \text{ and } R = \bigcup_{i \in I} \ker \pi_i.
\end{equation*} 
\end{definition}

If only a small finite number of groups is involved, for example only the two groups $G$ and $H$, we will write $G \ast H$ for their free product.

\begin{example}
The free product of the groups $G = \presentation{t}{t^2 = 1}$ and $H = \presentation{r}{r^3 = 1}$ is the group $G \ast H = \presentation{t,r}{t^2 = r^3 = 1}$ which we have already seen in Example \ref{ex_RTGroup}.
\end{example}

\begin{example}
Let $F_n$ and $F_m$ be the free groups generated by $n$ and $m$ elements respectively. Then the free product $F_n \ast F_m = F_{n+m}$ is the free group generated by $n+m$ elements.
\end{example}

\subsection{Basic linear groups}

In this section, we enumerate the definition of some more or less well-known basic groups, which will show up later from time to time throughout this document. We start with the general and special linear group.

\begin{definition}
\label{dfn_GenLinGrp}
\index{General linear group}
\index{GLnF@$\GLn{n}{F}$}
Let $F$ be a field and $n > 0$. The group of invertible $n$-by-$n$ matrices over  $F$ is called \emph{general linear group} and is denoted by
\begin{equation}
\label{eqn_GeneralLinearGroup}
\GLn{n}{F} := \setdef{M \in \SqMat{F}{n}}{\det M \ne 0}.
\end{equation}
\end{definition}

\begin{definition}
\label{dfn_SpLinGrp}
\index{Special linear group}
\index{SLnR@$\SLn{n}{R}$}
Let $R$ be a ring and $n > 0$. The group of $n$-by-$n$ matrices over $R$ having determinant 1, is called \emph{special linear group} and is denoted by
\begin{equation}
\label{eqn_SpecialLinearGroup}
\SLn{n}{R} := \setdef{M \in \SqMat{R}{n}}{\det M = 1}.
\end{equation}
Of course in the case, when $R$ is a field, the special linear group $\SLn{n}{R}$ is a subgroup of the general linear group $\GLn{n}{R}$. 
\end{definition}

\index{Group!center}
For both, the general and special linear group, one can construct the \emph{projective groups} by identifying matrices which differ by a scalar multiple. For this purpose, we define the \emph{center} of a group $G$ in the usual way:
\begin{equation}
\Center{G} := \setdef{z \in G}{zg = gz \quad \forall g \in G}.
\end{equation}
If $G$ is a matrix group, $\Center{G}$ consists precisely of all scalar multiples of the identity matrix within $G$, which is exactly what we need for this construction.

\begin{definition}
\label{dfn_ProjLinGrp}
\index{Projective!general linear group}
\index{PGLnF@$\PGLn{n}{F}$}
\index{Projective!special linear group}
\index{PSLnR@$\PSLn{n}{R}$}
As above, let $F$ be a field, $R$ a ring and $n > 0$. The \emph{projective general linear group} and the \emph{projective special linear group} are defined as
\begin{IEEEeqnarray}{rClCl}
\label{eqn_ProjGenLinGrp}
\PGLn{n}{F} &:=& \GLn{n}{F} &/& \Center{\GLn{n}{F}}, \IEEEyesnumber \\
\label{eqn_ProjSpLinGrp}
\PSLn{n}{R} &:=& \SLn{n}{R} &/& \Center{\SLn{n}{R}}. \IEEEyesnumber
\end{IEEEeqnarray}
\end{definition}

\begin{example}
\label{ex_ProjAndGenLinGrp}
If $M \in \GL{\R}$, then the corresponding equivalence class $[M]_{\sim} \in \PGL{\R}$ consists of all matrices $\lambda M$, $\lambda \in \R \setminus \{0\}$. If $\det(M) > 0$ then among these matrices, there is also a matrix with determinant 1, namely $M^{\prime} = \reci{\sqrt{\det(M)}} M$. Thus $M$ can also be thought of being a representative of the equivalence class $[M^{\prime}]_{\sim} \in \PSL{\R}$. In the other case, when $\det(M) < 0$, we cannot find such a matrix $M^{\prime}$, as the square root of a negative number is not real. Therefore $\PSL{\R}$ is a proper subgroup of $\PGL{\R}$,
\begin{equation*}
\PSL{\R} \lneq \PGL{\R}.
\end{equation*}
In contrast to that, if we take $M \in \GL{\C}$, then we can always set $M^{\prime} = \reci{\sqrt{\det(M)}}$ and identify $[M]_{\sim} \in \PGL{\C}$ with $[M]_{\sim} \in \PSL{\C}$ and we have
\begin{equation*}
\PSL{\C} \cong \PGL{\C}.
\end{equation*}
More generally, the groups $\PGLn{n}{F}$ and $\PSLn{n}{F}$ are isomorphic, if and only if $F$ is closed under taking the $n$-th root, otherwise $\PSLn{n}{F} \lneq \PGLn{n}{F}$.
\end{example}
